\documentclass{beamer}
\usepackage[utf8]{inputenc}
\usetheme{AMU}
\usepackage{amsmath, amssymb}
\DeclareMathOperator{\Var}{Var}
\DeclareMathOperator{\Cov}{Cov}
\usepackage{bm}
\usepackage{booktabs}

\title[Model Selection \& Optimisation]{Stochastic Models: Model Selection and Optimisation}
\subtitle{Lecture 6 \textendash{} From Evaluation to Decisions}
\author{Sebastian Müller}
\date{Lecture 6}

\newcommand{\definitionblock}[1]{\begin{block}{Definition}#1\end{block}}
\newcommand{\theorembox}[1]{\begin{block}{Theorem}#1\end{block}}
\newcommand{\notebox}[1]{\begin{block}{Note}#1\end{block}}
\newcommand{\ideabox}[1]{\begin{block}{Idea}#1\end{block}}

\begin{document}

\section{Overview}

\begin{frame}
  \titlepage
\end{frame}

\begin{frame}{Where We Are}
  \begin{itemize}
    \item \textbf{Lecture 1}: Motivation, Poisson refresher, simulation mindset.
    \item \textbf{Lecture 2}: Queueing basics, M/M/1 and M/M/$c$, Little's Law.
    \item \textbf{Lecture 3}: Error control, CLT and concentration, variance reduction, sensitivity.
    \item \textbf{Lecture 4}: Parameter estimation for Poisson, M/M/1, M/G/1; plug-in performance.
    \item \textbf{Lecture 5}: M/G/2 evaluation exercise from event log to fitted model.
  \end{itemize}
  \vspace{0.6em}
  \textit{Today: move from \emph{evaluation} of a given system to \emph{optimisation} of design choices.}
\end{frame}

\begin{frame}{Goals for Lecture 6}
  \begin{itemize}
    \item Decide which queueing model ($G/G/c/K/\dots$) is appropriate for a given dataset and question.
    \item Use fitted models to run simulation experiments with confidence intervals.
    \item Perform \textbf{scenario analysis}: how changes in load, variability, or capacity affect performance.
    \item Introduce \textbf{costs} and formulate a simple optimisation problem over design variables (e.g.\ number of servers).
    \item Connect the full pipeline: data $\to$ model $\to$ estimation $\to$ simulation $\to$ decision.
  \end{itemize}
\end{frame}

\section{Choosing a Model}

\begin{frame}{From Data to Candidate Models}
  \begin{itemize}
    \item Starting point: cleaned event log (Lecture 5) with arrivals, service times, and queue lengths.
    \item We want a \textbf{parsimonious} model that captures key features relevant for decisions.
    \item Two equally important dimensions:
    \begin{itemize}
      \item \textbf{Structural choice}: M/M/1, M/M/$c$, M/G/1, M/G/$c$, GI/G/$c$, finite buffers, abandonment, priorities, networks.
      \item \textbf{Parameter estimation}: $\hat\lambda$, service distribution parameters, variability measures.
    \end{itemize}
    \item Model choice should be driven by \emph{data}, \emph{domain knowledge}, and \emph{decision needs}.
  \end{itemize}
\end{frame}

\begin{frame}{Dimensions of Choice}
  \begin{itemize}
    \item \textbf{Arrival process}
      \begin{itemize}
        \item Poisson (homogeneous) vs.\ time-varying rate vs.\ overdispersed arrivals.
        \item Diagnostics: inter-arrival histogram, ECDF, overdispersion of counts, time-of-day effects.
      \end{itemize}
    \item \textbf{Service-time distribution}
      \begin{itemize}
        \item Exponential vs.\ deterministic offset + exponential vs.\ heavy-tailed.
        \item Diagnostics: empirical CV, skewness, log-survival plot, QQ-plot against exponential.
      \end{itemize}
    \item \textbf{System structure}
      \begin{itemize}
        \item Number of servers $c$, buffer capacity $K$, priority classes, reneging/abandonment.
        \item Single-node approximation vs.\ network / multi-stage system.
      \end{itemize}
  \end{itemize}
\end{frame}

\begin{frame}{Simple vs Complex Models}
  \ideabox{Start with a simple baseline model and add complexity only when necessary for decisions.}
  \begin{itemize}
    \item Baselines:
      \begin{itemize}
        \item M/M/$c$ for systems with no strong evidence against exponential assumptions.
        \item M/G/1 or M/G/$c$ when service-time variability clearly deviates from exponential.
      \end{itemize}
    \item More flexible options:
      \begin{itemize}
        \item GI/G/$c$ approximations when inter-arrivals also deviate from Poisson.
        \item Empirical service distributions (resampling) when no simple parametric family fits well.
      \end{itemize}
    \item Trade-off:
      \begin{itemize}
        \item Simpler models are easier to explain and calibrate.
        \item Richer models may better capture tails or bursts.
      \end{itemize}
  \end{itemize}
\end{frame}

\begin{frame}{Model Selection Guidelines}
  \small
  \begin{itemize}
    \item \textbf{Structural consistency}: data must not contradict assumptions (e.g.\ observed concurrency $\le c$ if you model $c$ servers; no abandonment if model excludes it).
    \item \textbf{Descriptive fit}: do simulated histograms / quantiles for arrivals and services match the data?
     \item \textbf{Performance fit}: does the model reproduce key metrics (mean $W_q$, $L_q$, utilisation) within uncertainty bands?
    \item \textbf{Decision relevance}: only add complexity if it changes the recommendation (e.g.\ number of servers to deploy).
    \item \textbf{Interpretability}: stakeholders should understand the assumptions and their limits.
  \end{itemize}
  \vspace{0.4em}
  \textit{Pick the simplest model that fits the data and supports the decision without violating observed behaviour.}
\end{frame}

\begin{frame}{Case Study Pitfall: Fitting M/G/3 Data with M/G/2}
  \begin{itemize}
    \item \textbf{What went wrong in Lecture 5?}
      \begin{itemize}
        \item Data inspection showed moments with 3 jobs in service $\Rightarrow$ evidence for $c=3$.
        \item Fitting M/G/2 underestimates capacity, inflates utilisation, and biases wait predictions.
      \end{itemize}
    \item \textbf{How to avoid it}
      \begin{itemize}
        \item Plot max number in service over time; compare to assumed $c$.
        \item Refit with $c=2$ vs $c=3$ and compare simulated $W_q$, $L_q$, tails \emph{with CIs}.
        \item Prefer the model that (i) matches observed service concurrency and (ii) does not degrade fit on waits/queues.
      \end{itemize}
    \item \textbf{Report explicitly} when structural choices are ambiguous; give a decision recommendation for each plausible $c$.
  \end{itemize}
\end{frame}

\section{Scenario Analysis}

\begin{frame}{What-If Analysis}
  \ideabox{Use the fitted model as a baseline and explore how performance changes when we perturb inputs or design choices.}
  \begin{itemize}
    \item Vary arrival rate: $\lambda = \alpha\,\hat\lambda$ for $\alpha \in \{0.8, 1.0, 1.2\}$ (demand uncertainty).
    \item Vary number of servers: $c \in \{1,2,3,4\}$ (staffing / capacity decisions).
    \item Optionally vary service variability: more/less variable $G$ with the same mean.
    \item For each scenario, simulate and compute CIs for $W_q$, $L_q$, utilisation, and service-level metrics.
  \end{itemize}
  \vspace{0.4em}
  \textit{Goal: identify regimes where the system is robust and regimes where it is fragile (near saturation).}
\end{frame}

\begin{frame}{Interpreting Scenario Tables}
  \begin{itemize}
    \item Present results as tables or heatmaps:
      \begin{itemize}
        \item Rows: number of servers $c$.
        \item Columns: demand multiplier $\alpha$ or other scenario parameter.
      \end{itemize}
    \item For each cell, report:
      \begin{itemize}
        \item Point estimates and 95\% CIs for $W_q$, $L_q$, utilisation.
        \item Possibly probability of violating a service-level target (e.g.\ $P(W_q > w^\ast)$).
      \end{itemize}
    \item Use these tables to communicate trade-offs:
      \begin{itemize}
        \item Where does adding a server significantly improve waiting times?
        \item Where do CIs for different configurations overlap (no clear winner)?
      \end{itemize}
  \end{itemize}
\end{frame}

\section{Cost and Optimisation}

\begin{frame}{Adding a Cost Model}
  \definitionblock{We translate performance metrics into monetary (or utility) costs to support design decisions.}
  \begin{itemize}
    \item Basic ingredients:
      \begin{itemize}
        \item Server cost $c_{\text{server}}$ per server per unit time.
        \item Waiting cost $c_{\text{wait}}$ per unit waiting time per job.
        \item Optional penalty for SLA violations (e.g.\ if $W_q > w^\ast$).
      \end{itemize}
    \item Given arrival rate $\lambda$ and mean waiting time $W_q(c)$ for configuration $c$:
      \[
        C(c) = c_{\text{server}} \cdot c \;+\; c_{\text{wait}} \cdot \lambda\, W_q(c).
      \]
    \item $W_q(c)$ is estimated via simulation; hence $C(c)$ is also stochastic with a CI.
  \end{itemize}
\end{frame}

\begin{frame}{Simulation-Based Optimisation}
  \begin{itemize}
    \item For small discrete design spaces (e.g.\ $c=1,\dots,6$), we can simply enumerate options:
      \begin{itemize}
        \item For each $c$, estimate $W_q(c)$ via multiple replications.
        \item Compute $\hat C(c)$ and a 95\% CI using the CI for $W_q(c)$.
      \end{itemize}
    \item Choose the configuration with the smallest $\hat C(c)$, but report uncertainty:
      \begin{itemize}
        \item If CIs for $C(c)$ overlap, it may be safer to report a set of near-optimal options.
      \end{itemize}
    \item Optional: add constraints, e.g.\ $P(W_q > w^\ast) \le \alpha$ (service-level constraint), and pick the cheapest feasible $c$.
  \end{itemize}
\end{frame}

\begin{frame}{From Numbers to Recommendations}
  \begin{itemize}
    \item Translate optimisation results into plain-language recommendations:
      \begin{itemize}
        \item ``With current demand, 2 agents minimise expected cost; 3 agents only improve waiting times slightly.''
        \item ``If demand increases by 20\%, the system becomes unstable with 2 agents; we recommend adding a third.''
      \end{itemize}
    \item Emphasise uncertainty: use CIs and scenario analysis to show robustness.
    \item Document assumptions: model structure, stationarity, independence, stability conditions.
  \end{itemize}
\end{frame}

\section{Wrap-Up}

\begin{frame}{Course Pipeline Recap}
  \begin{itemize}
    \item \textbf{Model}: choose a queueing framework and assumptions informed by data and context.
    \item \textbf{Estimate}: infer parameters from data (Lecture 4) and quantify input uncertainty.
    \item \textbf{Simulate}: implement discrete-event simulations with error control (Lecture 3).
    \item \textbf{Evaluate}: compare model predictions to observed performance (Lecture 5).
    \item \textbf{Optimise}: use costs and constraints to recommend configurations (Lecture 6).
  \end{itemize}
  \vspace{0.4em}
  \textit{This is the workflow you should apply in your final projects.}
\end{frame}

\begin{frame}{For Your Projects}
  \begin{itemize}
    \item Clearly state:
      \begin{itemize}
        \item Modelling assumptions and chosen queueing framework.
        \item How parameters were estimated and with what uncertainty.
        \item How simulation was used to compare policies or configurations.
        \item What decision or recommendation you make and how robust it is.
      \end{itemize}
    \item Focus on a coherent modelling story rather than perfectly tuned parameters.
  \end{itemize}
\end{frame}

\end{document}
