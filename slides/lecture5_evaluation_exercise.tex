\documentclass{beamer}
\usepackage[utf8]{inputenc}
\usetheme{AMU}
\usepackage{amsmath, amssymb}
\DeclareMathOperator{\Var}{Var}
\DeclareMathOperator{\Cov}{Cov}
\usepackage{bm}
\usepackage{booktabs}

\title[Evaluation Exercise]{Stochastic Models: Evaluation Exercise}
\subtitle{Lecture 5 \textendash{} M/G/2 Case Study}
\author{Sebastian Müller}
\date{Lecture 5}

\newcommand{\definitionblock}[1]{\begin{block}{Definition}#1\end{block}}
\newcommand{\theorembox}[1]{\begin{block}{Theorem}#1\end{block}}
\newcommand{\notebox}[1]{\begin{block}{Note}#1\end{block}}
\newcommand{\ideabox}[1]{\begin{block}{Idea}#1\end{block}}

\begin{document}

\section{Overview}

\begin{frame}
  \titlepage
\end{frame}

\begin{frame}{Goal of the Evaluation}
  \begin{itemize}
    \item Work from a raw event log to a plausible stochastic queueing model.
    \item Infer arrival and service parameters from data, including uncertainty.
    \item Assess performance (waiting times, mean number in queue $L_q$, utilisation) of a two-server system.
    \item Practice telling a clear modelling story from noisy real-world data.
  \end{itemize}
\end{frame}

\section{Scenario}

\begin{frame}{System Description}
  \begin{itemize}
    \item Small ML-backed support platform handling ``complex'' tickets.
    \item Two identical human agents process tickets in parallel (two servers).
    \item All incoming tickets during the observation window enter this two-server system and are recorded in the log.
    \item Each complex ticket incurs a fixed overhead (reading context, loading tools) plus a random processing time.
  \end{itemize}
  \medskip
  \notebox{
  For modelling we treat this as an $M/G/2$ queue: Poisson arrivals (unknown rate), i.i.d.\ service times with unknown distribution $G$, and two identical servers.
  }
\end{frame}

\begin{frame}{What Data You Get}
  \begin{itemize}
    \item One CSV log from a contiguous observation window (no gaps).
    \item One row per completed ticket, with fields:
      \begin{itemize}
        \item arrival\_time, start\_service\_time, completion\_time
        \item service\_time, wait\_time, system\_time
        \item queue\_len\_at\_arrival (tickets in system just before arrival)
      \end{itemize}
    \item Jobs that arrive before the end of the window are included even if they complete later.
  \end{itemize}
\end{frame}

\section{Tasks}

\begin{frame}{Modelling Tasks}
  \begin{itemize}
    \item Clean and validate the log (nonnegative waits, temporal ordering).
    \item Diagnose the arrival process: inter-arrival distribution, approximate stationarity, estimate $\hat\lambda$ with a CI.
    \item Explore the service-time distribution: evidence of a lower bound (fixed overhead) and an approximately memoryless tail.
    \item Propose a simple parametric family for $G$ (e.g.\ constant offset + exponential) and fit its parameters.
  \end{itemize}
\end{frame}

\begin{frame}{Performance and Uncertainty}
  \begin{itemize}
    \item Use your fitted model to estimate utilisation $\hat\rho = \hat\lambda \hat m_1 / 2$ for the two-server system.
    \item Estimate the mean number of waiting jobs $L_q$ from the data and relate it to $\hat\lambda$ and $\overline{W}_q$ (Little's Law).
    \item Compare empirical mean waiting time from the log to a model-based prediction (via simulation or an $M/G/2$ approximation).
    \item Quantify uncertainty for at least one key metric (e.g.\ bootstrap or regenerative analysis over cycles).
  \end{itemize}
  \medskip
  \notebox{
  You may reuse any error-control or bootstrap techniques from earlier lectures (delta method, input bootstrap, regenerative bootstrap, simulation-based CIs, \dots).
  }
\end{frame}

\section{Deliverables}

\begin{frame}{Deliverables and Grading}
  \begin{itemize}
    \item A clear description of your chosen model (arrival process, service family, number of servers).
    \item Parameter estimates with at least one uncertainty measure (CI or standard error).
    \item Comparison between empirical and model-based performance (focus on waiting time, mean number in queue $L_q$, and utilisation).
    \item Short discussion of diagnostics and model limitations (what the model misses, robustness of conclusions).
  \end{itemize}
  \medskip
  \notebox{
  Emphasis is on \emph{reasoned modelling and diagnostics}, not on guessing the exact hidden parameters.
  }
\end{frame}

\end{document}
